\documentclass[a4paper, 11pt]{article}
\usepackage[ngerman]{babel}
\usepackage[utf8]{inputenc}
\usepackage{fullpage}
\usepackage{amsmath}
\usepackage{graphicx}

\title{\textbf{Aufgabe 2 - Radiale Basisfunktionen}}
\author{Gruppe AC}

\begin{document}
\maketitle

\section*{Teilaufgabe 1}
Mit $\alpha=-1$ gilt folgende Gleichung für $h(r)$
\[
h(r) = h( \left\lVert x_i - x_j \right\rVert) = \left\lVert x_i - x_j \right\rVert ^2 + 0.001
\]
Es ergibt sich folgende Matrix H:
\begin{align*}
H &= \left( \begin{array}{ccc}
h( \left\lVert x_1 - x_1 \right\rVert) & h( \left\lVert x_1 - x_2 \right\rVert) & h( \left\lVert x_1 - x_2 \right\rVert) \\
h( \left\lVert x_2 - x_1 \right\rVert) & h( \left\lVert x_2 - x_2 \right\rVert) & h( \left\lVert x_2 - x_3 \right\rVert) \\
h( \left\lVert x_3 - x_1 \right\rVert) & h( \left\lVert x_3 - x_2 \right\rVert) & h( \left\lVert x_3 - x_3 \right\rVert)
\end{array} \right) \\
 &= \left( \begin{array}{ccc}
0.001 & 1.001 & 4.001 \\
1.001 & 0.001 & 1.001 \\
4.001 & 1.001 & 0.001
\end{array} \right) \\
\end{align*}


\section*{Teilaufgabe 2}
\begin{align*}
	T_3 &= g(x_3) = \sum_{i=1}^3 w_i \cdot h(\left\lVert x_i - x_3 \right\rVert) \\
	    &= w_1 \cdot h(\left\lVert x_1 - x_3 \right\rVert) + w_2 \cdot h(\left\lVert x_2 - x_3 \right\rVert) + w_3 \cdot h(\left\lVert x_3 - x_3 \right\rVert) \\
	w_3 &= \frac{T_3 - w_1 \cdot h(\left\lVert x_1 - x_3 \right\rVert) - w_2 \cdot h(\left\lVert x_2 - x_3 \right\rVert)}{h(\left\lVert x_3 - x_3 \right\rVert)} \\
		&= \frac{-1 - 1.752 \cdot 4.001 + 8.004 \cdot 1.001}{0.001} \\
		& = 2,252
\end{align*}

\section*{Teilaufgabe 3}
a)
\begin{align*}
	g(3) &= \sum_{i=1}^3 w_i \cdot h(\left\lVert x_i - 3 \right\rVert) \\
		 &= 1.752 \cdot h(\left\lVert 0 - 3 \right\rVert) - 8.004 \cdot h(\left\lVert 1 - 3 \right\rVert) + 2.252 \cdot h(\left\lVert 2 - 3 \right\rVert) \\
		 &= - 14 
\end{align*}
b) Handelt es sich um einen interpolierten Funktionswert? Nein, $x=3$ befindet sich nicht innerhalb des Intervalls $[x_1, x_3]=[0,2]$ und ist somit kein interpolierter Funktionswert.

\section*{Teilaufgabe 4}
\begin{enumerate}
	\item[$\alpha = -1$] gehört zu Grafik B. Die Formel für $h(r)$ wird zu einer Parabelgleichung.
	\item[$\alpha = -0.3$] gehört zu Grafik C. Durch den alpha-Wert muss die Kurve der Funktion $h(r)$ nach oben geöffnet sein.  
	\item[$\alpha = 0.15$] gehört zu Grafik A. Durch den positiven alpha-Wert muss die Kurve der Funktion $h(r)$ nach unten geöffnet sein.
\end{enumerate}


\end{document}